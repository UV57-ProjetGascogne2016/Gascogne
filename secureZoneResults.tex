
\paragraph{Répartition des bateaux}
Dans un premier temps, nous avions penser à répartir les bateaux avec un espacement régulier selon l'abscisse curviligne de l'ellipse.
Cette méthode étant finalement très difficile à mettre en place d'un point de vue mathématique et algorithmique, cette solution a été abandonnée. Il aurait fallu, pour se faire, approximer les intégrales de Wallis par un développement limité assez complexe, ce qui implique un temps de programmation important et un temps de calcul beaucoup trop long pour une simple étude de faisabilité.
C'est pour cela que nous avons décidé de nous orienter vers un retard angulaire sur chacun des robots. Ainsi, chaque robot suit le robot précédent avec un retard angulaire fixe dépendant du nombre de robots parcourant l'ellipse. Cela implique que les robots vont être d'autant plus proches les uns des uns qu'il sont près des apogées de l'ellipse, et d'autant plus éloignés aux périgées.
Néanmoins, malgré la non-linéarité du placement des robots, nous n'avons pas de rupture de la zone de détection. En effet, comme les robots vont plus vite aux périgées, la trace qu'ils "laissent" est plus grande, et inversement aux apogées. De ce fait, on peut garantir la sécurité de la zone à surveiller, même si les robots ne sont pas régulièrement espacés.


