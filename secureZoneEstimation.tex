\subsection{Basics and Usefulness of Interval Arithmetics}
Normal solver equation are efficient with linear sytems but in a real environment 
\subsection{Thick Functions}
Introduce the concept of MAYBE
\subsection{Special Test for Gascogne Surveillance}
We found a test to determine if a box is inside the secure zone covered 
by one


\begin{algorithm}
\caption{Is $\mathbf{X} \subseteq \mathbb{S}$ , $\mathbb{S} =$ Secure Zone and $\mathbf{X} \in \mathbb{R^{\textnormal{\ensuremath{2}}}}$ }
\begin{algorithmic}
\REQUIRE $X$,$range $ (reach of boat),$m$ (position of boat)
\STATE $Xmx \leftarrow max([0,0], sign( (X[0]-m[0].ub())*(X[0]-m[0].lb()))) * \
              min((X[0]-m[0].lb())^{\textnormal{\ensuremath{2}}},(X[0]-m[0].ub())^{\textnormal{\ensuremath{2}}} )$
\STATE $ Xmy \leftarrow max([0,0], sign( (X[1]-m[1].ub())*(X[1]-m[1].lb()))) * min((X[1]-m[1].lb())^{\textnormal{\ensuremath{2}}},(X[1]-m[1].ub())^{\textnormal{\ensuremath{2}}} )$
\STATE $Xm \leftarrow Xmx + Xmy $
\STATE $Xp \leftarrow max((X[0]-m[0].lb())^{\textnormal{\ensuremath{2}}},(X[0]-m[0].ub())^{\textnormal{\ensuremath{2}}}) + \
                  max((X[1]-m[1].lb())^{\textnormal{\ensuremath{2}}},(X[1]-m[1].ub())^{\textnormal{\ensuremath{2}}})$
\STATE $Xub \leftarrow Xp | Xm$
\IF{$Xub \cap \mathbb{[ \,\textnormal{0},\textnormal{range}^{\textnormal{\ensuremath{2}}}] \,} =  \emptyset $}
\RETURN $\textnormal{OUT}$
\ELSIF {$Xub \subseteq \mathbb{[ \,\textnormal{0},\textnormal{range}^{\textnormal{\ensuremath{2}}}] \,} $}
\RETURN $\textnormal{IN}$
\ELSE
\IF{ $\textnormal{range}^{\textnormal{\ensuremath{2}}} - Xp.ub() < \textnormal{0}] $}
\IF{ $Xm - \textnormal{range}^{\textnormal{\ensuremath{2}}} \subseteq \mathbb{[ \,-\infty,\textnormal{0}] \,}$}
\RETURN $\textnormal{MAYBE}$
\ELSE
\RETURN $\textnormal{UNKNOWN2}$
\ENDIF
\ELSE
\RETURN $\textnormal{UNKNOWN}$
\ENDIF
\ENDIF
\end{algorithmic}
\end{algorithm}

Differente result In testing : \newline

\begin{center}
\begin{tabular}{|m{0.10\linewidth}|m{0.5\linewidth}|}
\hline
 Symbol & Meaning  \\ \hline
 0 & Box is not in the Secure Zone  \\ \hline
 1 &  Box is in the secure zone \\ \hline
 ? &  Box may be in the secure zone  \\ \hline
[0,?] & Box may be in the secure zone \\ \hline
[?,1] & Box may be in the secure zone\\ \hline
[0,1] & Box may be in the secure zone or out \\ \hline
    
   
\end{tabular}
\end{center}


\begin{center}
\begin{tabular}{|m{0.10\linewidth}|m{0.07\linewidth}|m{0.07\linewidth}|m{0.07\linewidth}|m{0.07\linewidth}|m{0.07\linewidth}|m{0.07\linewidth}|}
\hline
Test1/Test2 & 0 & 1 & ? & [0,?] &  [?,1] & [0,1] \\ \hline
          0 & 0 & 1 & ? & [0,?] &  [?,1] & [0,1]  \\ \hline
          1 &   & 1 & 1 &   1   &    1   &   1  \\ \hline
          ? &   &   & ? &   ?   &  [?,1] & [?,1] \\ \hline
      [0,?] &   &   &   & [0,?] &  [?,1] & [0,1] \\ \hline
      [?,1] &   &   &   &       &  [?,1] & [?,1] \\ \hline
      [0,1] &   &   &   &       &        & [0,1]  \\ \hline
    
   
\end{tabular}
\end{center}

\begin{algorithm}
\caption{Is $\mathbf{X} \subseteq \mathbb{S}$ , $\mathbb{S} =$ Secure Zone and $\mathbf{X} \in \mathbb{R^{\textnormal{\ensuremath{2}}}}$ }
\begin{algorithmic}
\REQUIRE $X$,$range $ (reach of a boat),$\{m\}$ (list of position of boats)
\STATE $R \leftarrow \textnormal{Algorithme1(X,range,\{m\})}$
\IF{$\textnormal{IN} \subset R$}
\RETURN $\textnormal{IN}$
\ELSIF {$\textnormal{UNKNOWN} \subset R $}
\RETURN $\textnormal{UNKNOWN}$
\ELSIF {$\textnormal{MAYBE} \subset R $}
\RETURN $\textnormal{MAYBE}$
\ELSIF {$\forall r \in R $, $ r = \textnormal{OUT} $}
\RETURN $\textnormal{OUT}$
\ELSE
\RETURN $\textnormal{UNKNOWN}$
\ENDIF
\end{algorithmic}
\end{algorithm}
