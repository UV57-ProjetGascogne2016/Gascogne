\documentclass[10pt,a4paper]{report}
\usepackage[utf8]{inputenc}
\usepackage[left=2cm,right=2cm,top=2cm,bottom=2cm]{geometry}
\usepackage{amsmath}
\usepackage{amsfonts}
\usepackage{amssymb}
\usepackage{color}
\usepackage{array}
\usepackage{graphicx}
\usepackage{caption} 
\usepackage{graphicx}
\usepackage[french]{babel}
\usepackage{hyperref}
\usepackage{algorithm}
\usepackage{algorithmic}

\begin{document}
\title{\textbf{ {\Huge Surveillance du Golfe de Gascogne}}}

\maketitle

\pagebreak

\begin{LARGE}
\textbf{Introduction}
\end{LARGE}



\bigskip

Le but de ce projet est de détecter des intrus dans une zones maritime donnée. Ici, nous allons nous intéressé au Golfe de Gascogne. Nous allons donc utiliser un groupe de robots équipés de GPS et nous allons implémenter des algorithmes et un stratégie pour assurer la sureté d'une zone. 

\bigskip

Nous allons avoir plusieurs hypothèses de départ:
\begin{itemize}
\item Nous supposeront que l'étude peut être simulée dans un monde en deux dimensions.
\item Les intrus auront une vitesse constante tout au long de la simulation.
\item Tous les robots pourront voir autour d'eux sur une distance $d_{i}$ et détecter si  un intrus est présent sur cette zone.
\item Nous définissons une zone sûre comme un ensemble de points où un intrus ne peut se trouver.
\end{itemize}

\bigskip

Nous allons donc présenter au final deux livrables:
\begin{itemize}
\item Une simulation théorique faite en grande partie en langage python qui permettra de déterminer la pertinence de nos algorithmes.
\item Un simulation sur des robots chars disponibles au club robotique de l'ENSTA Bretagne.
\end{itemize}

\chapter{Quelles stratégies ont été retenues}

Expliquer comment nous allons répondre aux exigences

1) Description de la trajectoire retenue
2) Méthode retenues (OpenCV / théorie des intervalles )


\chapter{Mise en œuvre des solutions}

\paragraph{Trajectoire elliptique}

\paragraph{Répartition des bateaux}
Dans un premier temps, nous avions penser à répartir les bateaux avec un espacement régulier selon l'abscisse curviligne de l'ellipse.
Cette méthode étant finalement très difficile à mettre en place d'un point de vue mathématique et algorithmique, cette solution a été abandonnée. Il aurait fallu, pour se faire, approximer les intégrales de Wallis par un développement limité assez complexe, ce qui implique un temps de programmation important et un temps de calcul beaucoup trop long pour une simple étude de faisabilité.
C'est pour cela que nous avons décidé de nous orienter vers un retard angulaire sur chacun des robots. Ainsi, chaque robot suit le robot précédent avec un retard angulaire fixe dépendant du nombre de robots parcourant l'ellipse. Cela implique que les robots vont être d'autant plus proches les uns des uns qu'il sont près des apogées de l'ellipse, et d'autant plus éloignés aux périgées.
Néanmoins, malgré la non-linéarité du placement des robots, nous n'avons pas de rupture de la zone de détection. En effet, comme les robots vont plus vite aux périgées, la trace qu'ils "laissent" est plus grande, et inversement aux apogées. De ce fait, on peut garantir la sécurité de la zone à surveiller, même si les robots ne sont pas régulièrement espacés.


\section{Évolution de la consigne au cours du temps}

Alice / Eric / Chuch

\section{Regulation of multiples real robots for demonstration of concept}


\section{Regulation of the robot pack}

Two methods have been studied and implemented to regulate the pack of robots. The first is a method by artificial potential field, robust and easy to debug. This method was chosen to regulate the real robots on the play field. The second method is a method of looping linearisation, very efficient. This method was chosen for the simulation because its integration in a simulation is easier than on robots.

\subsection{Artificial potential field}
Let $p_{robot}$ be the position of our considered robot, let $p_{target}$ and $v_{target}$ be the position and speed of the target we want the robot to reach.
We can consider the robot and the target as two particles of opposite charge, and then compute the potential field between them. In case of obstacles, we can consider them as particles of same charge than the robot's.
The potential field method calculate the instantaneous speed vector $w(p_{robot},t)$ the robot need to reach (or at least follow) the target. To compute that speed, we use the potential $V$ between the robot and the target :\\
\[ V(p_{robot}) = v^T_{target}. p_{robot} + \|p_{robot}-p_{target}\|^2 \]
And compute the gradient of that potential to find the order $w(p_{robot},t)$:
\[w(p_{robot},t) = -grad(V(p_{robot})) = -\frac{dV}{dP}(p)^T\]
So :
\[w(p_{robot},t) = v_{target}-2.(p_{robot}-p_{target})\]
For w, we compute the objective speed and course:
\[\bar{v} = |w\| \]
\[\bar{\theta} = tan(\frac{w_y}{w_x})\]

Alice / Eric / Chuch ou Totoboule / Sylvain selon la solution retenue

\section{Description du fonctionnement du simulateur}

Totoboule / Sylvain + Image à faire

\section{Secure zone estimation with interval analysis}

\subsection{Basics and Usefulness of Interval Arithmetics}
 
Normal solver equation are efficient with linear sytems but in a real environment 

\subsection{Thick Functions}

\subsection{Special Test for Gascogne Surveillance}

\begin{algorithm}
\caption{Is $\mathbf{X} \subseteq \mathbb{S}$ , $\mathbb{S} =$ Secure Zone and $\mathbf{X} \in \mathbb{R^{\textnormal{\ensuremath{2}}}}$ }
\begin{algorithmic}
\REQUIRE $range \geq 0 \vee \mathbf{X} \neq \emptyset$
\STATE $Xmx \leftarrow max([0,0], sign( (X[0]-m[0].ub())*(X[0]-m[0].lb()))) * \
              min((X[0]-m[0].lb())^{\textnormal{\ensuremath{2}}},(X[0]-m[0].ub())^{\textnormal{\ensuremath{2}}} )$
\STATE $ Xmy \leftarrow max([0,0], sign( (X[1]-m[1].ub())*(X[1]-m[1].lb()))) * min((X[1]-m[1].lb())^{\textnormal{\ensuremath{2}}},(X[1]-m[1].ub())^{\textnormal{\ensuremath{2}}} )$
\STATE $Xm \leftarrow Xmx + Xmy $
\STATE $Xp \leftarrow max((X[0]-m[0].lb())^{\textnormal{\ensuremath{2}}},(X[0]-m[0].ub())^{\textnormal{\ensuremath{2}}}) + \
                  max((X[1]-m[1].lb())^{\textnormal{\ensuremath{2}}},(X[1]-m[1].ub())^{\textnormal{\ensuremath{2}}})$
\STATE $Xub \leftarrow Xp | Xm$
\IF{$Xub \cap \mathbb{[ \,\textnormal{0},\textnormal{range}^{\textnormal{\ensuremath{2}}}] \,} =  \emptyset $}
\RETURN $\textnormal{OUT}$
\ELSIF {$Xub \subseteq \mathbb{[ \,\textnormal{0},\textnormal{range}^{\textnormal{\ensuremath{2}}}] \,} $}
\RETURN $\textnormal{IN}$
\ELSE
\IF{ $\textnormal{range}^{\textnormal{\ensuremath{2}}} - Xp.ub() < \textnormal{0}] $}
\IF{ $Xm - \textnormal{range}^{\textnormal{\ensuremath{2}}} \subseteq \mathbb{[ \,-\infty,\textnormal{0}] \,}$}
\RETURN $\textnormal{MAYBE}$
\ELSE
\RETURN $\textnormal{UNKNOWN2}$
\ENDIF
\ELSE
\RETURN $\textnormal{UNKNOWN}$
\ENDIF
\ENDIF
\end{algorithmic}
\end{algorithm}

Differente result In testing : \newline

\begin{center}
\begin{tabular}{|m{0.10\linewidth}|m{0.5\linewidth}|}
\hline
 Symbol & Meaning  \\ \hline
 0 & Box is not in the Secure Zone  \\ \hline
 1 &  Box is in the secure zone \\ \hline
 ? &  Box may be in the secure zone  \\ \hline
[0,?] & Box may be in the secure zone \\ \hline
[?,1] & Box may be in the secure zone\\ \hline
[0,1] & Box may be in the secure zone or out \\ \hline
    
   
\end{tabular}
\end{center}


\begin{center}
\begin{tabular}{|m{0.10\linewidth}|m{0.07\linewidth}|m{0.07\linewidth}|m{0.07\linewidth}|m{0.07\linewidth}|m{0.07\linewidth}|m{0.07\linewidth}|}
\hline
Test1/Test2 & 0 & 1 & ? & [0,?] &  [?,1] & [0,1] \\ \hline
          0 & 0 & 1 & ? & [0,?] &  [?,1] & [0,1]  \\ \hline
          1 &   & 1 & 1 &   1   &    1   &   1  \\ \hline
          ? &   &   & ? &   ?   &  [?,1] & [?,1] \\ \hline
      [0,?] &   &   &   & [0,?] &  [?,1] & [0,1] \\ \hline
      [?,1] &   &   &   &       &  [?,1] & [?,1] \\ \hline
      [0,1] &   &   &   &       &        & [0,1]  \\ \hline
    
   
\end{tabular}
\end{center}


\section{Prise en compte du passé grâce à OpenCV}

Maël / Khadim / David  ... A vous de choisir vos sous-sections

\subsubsection{Pavage des images}

\subsubsection{Érosion des zones de sureté}


\chapter{Réalisation de test sur robot char}
Benoit / Maxime / Pierre ... A vous de choisir vos sections

Pour la réalisation d'une simulation sur des robots, nous avons du implémenter un architecture afin de permettre à nos programme de gérer d'une part nos robots et d'autre part de voir si le résultat théorique correspond aux résultats expérimentaux.

\medskip

Nous aboutissons au final à l'architecture suivante:

\bigskip

\begin{figure}[ht]
\centering
    \includegraphics[scale=0.8,angle=0]{SyntheseExp.png}
    \caption{Architecture mise en place pour la simulation sur robot char.}
    \label{fig:SyntheseExp}
\end{figure}

\bigskip

\section{Matériel}
\subsection{GPS}
ici la description des GPS utilisés
\subsection{Robot}
ici la description des robots et des récepteurs
\section{Analyse des trames GPS}
\textcolor{blue}{\textit{Pierre JACQUOT}}
Afin de récupérer les coordonnées GPS des robots nous avons décidé d'utiliser la tram \textit{GPRMC} envoyée par les émetteurs GPS. L'exemple ci-dessous montre l'aspect d'une trame : \newline
\begin{center}
\begin{tabular}{|m{0.05\linewidth}|m{0.07\linewidth}|m{0.07\linewidth}|m{0.07\linewidth}|m{0.08\linewidth}|m{0.07\linewidth}|m{0.07\linewidth}|m{0.04\linewidth}|m{0.07\linewidth}|m{0.1\linewidth}|m{0.07\linewidth}|}
\hline
    081836 & A & 3751.65 & S & 14507.36 & E & 000.0 & 360.0 & 130998,01 & 011.3 & E*62  \\ \hline
    heure UTC & Status des Donneés & Latitude & N ou S & Longitude & E ou W & Vitesse en nœuds & Cap & Date UT & Déviation Magnétique & E ou W et Checksum \\ \hline

\end{tabular}
\end{center}
Ici les données importantes sont la longitude, la latitude le temps et la vitesse au niveau du sol. Afin de les récupérer,nous avons utilisé le package python pynmea, qui permet de facilement récupérer et parser une trame GPS. \newline
La latitude et la longitude récupérées ont ici une structure particulière qu'il est nécessaire de modifier afin d'avoir des données compréhensibles et facilement exploitables. L'exemple ci-dessous montre la structure de la longitude et de la latitude. \newline
\begin{itemize}
  \item Longitude  : 12311.12,W   Longitude 123 degrée. 11.12 min Ouest
  \item Latitude : 4916.45,N    Latitude 49 degrée 16.45 min Nord
\end{itemize}

Les directions cardinales associées la latitude (Nord et Sud) et à la longitude (Est et Ouest) permettent de savoir respectivement de quel côte du méridien de Greenwich ou de l'équateur nous nous trouvons. En l'occurrence, nous aurons à Brest une latitude "négative" et une longitude positive.\newline
D'un point de vue programmation, il a donc était nécessaire de convertir la longitude et la latitude en degré et de prendre en compte les directions cardinales afin de rendre négative ou non la longitude et latitude.\newline
En plus de la latitude et de la longitude, notre programme permet aussi de récupérer les coordonnées UTM (Universal Transverse Mercator). Ces coordonnées permettent de représenter la position des robots dans un repère local et peuvent être plus facile à manipuler que des longitudes et latitudes car ces coordonnées sont simplement sous la forme de nombres décimaux. Le système UTM se base sur une division du monde en secteur, et permet ainsi d'obtenir des coordonnées locales pour chaque secteur. Les secteurs découpants la France sont décrits dans la figure ci-dessous :\newline
\begin{center}
\includegraphics[scale=0.2]{secteurUTM.png} 
\captionof{figure}{Secteur UTM}
\label{fig1}
\end{center}
On constate à l'aide de la figure précédente que nous nous trouvons dans le secteur 30. Cependant afin d'effectuer ses projections par secteur, il faut aussi choisir un système géodésique représentant la Terre. Nous avons choisi le système le plus courant pour ce projet, soit le système WGS-84.

\section{Structure du code}

Quatre scripts python: GPS2.py, commande.py, testserial.py et ServeurN.py

\section{Résultats}



\end{document}
