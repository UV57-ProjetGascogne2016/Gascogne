\documentclass[10pt,a4paper]{report}
\usepackage[utf8]{inputenc}
\usepackage[left=2cm,right=2cm,top=2cm,bottom=2cm]{geometry}
\usepackage{amsmath}
\usepackage{amsfonts}
\usepackage{amssymb}
\usepackage{graphicx}
\usepackage[french]{babel}
\usepackage{hyperref}


\begin{document}
\title{\textbf{ {\Huge Surveillance du golfe de Gascogne}}}

\maketitle

\pagebreak

\chapter{Description des objectifs }

description succinctes des exigences demandées et des différents problématiques du projet.


\chapter{Quelles stratégies ont été retenues}

Expliquer comment nous allons répondre aux exigences

1) description de la trajectoire retenue
2) Méthode retenues (OpenCV / théorie des intervalles )


\chapter{Mise en œuvre des solutions}

\section{Réalisation d'une trajectoire elliptique et répartition des bateaux}

Alice / Eric / Chuch

\section{Évolution de la consigne au cours du temps}

Alice / Eric / Chuch

\section{Régulation de la meute de robot}

Alice / Eric / Chuch ou Totoboule / Sylvain selon la solution retenue

\section{Description du fonctionnement du simulateur}

Totoboule / Sylvain + Image à faire

\section{Estimation de la zone de sureté grâce à la théorie des intervalles}

Alice / Eric / Chuch / Elouan ... au choix mais je pense que Elouan est le plus connaisseur ;)


\section{Prise en compte du passé grâce à OpenCV}

Maël / Khadim / David  ... A vous de choisir vos sous-sections

\subsubsection{Pavage des images}

\subsubsection{Érosion des zones de sureté}


\chapter{Réalisation de test sur robot char}

Benoit / Maxime / Pierre ... A vous de choisir vos sections

\end{document}